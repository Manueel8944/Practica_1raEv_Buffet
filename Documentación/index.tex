\documentclass[a4paper,12pt]{article}

\usepackage[utf8]{inputenc}
\usepackage[T1]{fontenc}
\usepackage[spanish]{babel}
\usepackage{hyperref}
\usepackage{enumitem}
\usepackage{tabularx}
\usepackage{float}
\usepackage{longtable}
\usepackage{graphicx}

\title{Documentación Aplicación Buffet}
\author{Manuel Añel García}
\date{\today}

\begin{document}

\maketitle

\section{Tipo de datos}

    \begin{itemize}
        \item Este es uno de los principales fallos que tuve ya que en mongo los numeros son Number y yo apartir de eso le puse Number en el Backend y las Clases entonces en la practica al intentar igualar o diferenciar con un if un tipo Number con un Int no me dejaba.

        \begin{figure}[H]
            \centering
            \includegraphics[width=0.5\linewidth]{image.png}
        \end{figure}
        
        \item Tambien para los tipo de datos de los TextView, si quiero ponerle un numero tengo que transformarlo a string.

        \begin{figure}[H]
            \centering
            \includegraphics[width=0.7\linewidth]{image1.png}
            \label{fig:placeholder}
        \end{figure}
    \end{itemize}



\section{Llamadas retrofit}
    Esta es básicamente el principal error y más importante que he tenido, en comparacion con JavaFX en Android estudio usando retrofit no podía hacer dos llamadas HTTP en la misma función y para soulcionarlo está todo muy distribuido en funciones y se puede ver que casi todas las funciones tienen otras funciones dentro con parametros para usar los datos de una llamada HTTP en otra función.

    

\section{Pasar los proyectos a casa}
    Estuve mucho tiempo pasando los proyectos de clase a casa ya que no se me ponía bien el JDK o no me leía el pom etc...

\section{Hacer cambios en el UI JavaFX}
Se me complicó para en las llamadas HTTP cambiar la UI de la aplicación ya que están en un hilo sitinto al principal, así que es necesario usar Platform.runLater(() -> {})

\section{Contadores con Lambdas}
Esto me llevo un tiempo porque Platform.runLater(() -> {}) es una lambda entonces no me dejaba hacer un contador fuera de un bucle y dentro despues de cada recorrido sumarlo si estaba la lambda dentro del bucle, me ponía que no era un parametro final, osea que iba cambiando y dentro del bucle tuve que relacionar el parametro contador a otro parametro que sea final.
\begin{figure}[H]
    \centering
    \includegraphics[width=1\linewidth]{image2.png}
\end{figure}

\section{ID en fxml}
Este fue un error que no me dejaba modificar los elementos en JavaFX ya que cuando hice la vista con el SceneBuilder puse el id en un parametro que había, y se ponia como "id=1" y eso no me salia como error pero luego al ejecutarlo con los cambios de colores del controller no me cambiaba nada y buscando me di cuenta que para que se refieran en el controller el id tiene que ser "fx:id=1"

\begin{figure}[H]
    \centering
    \includegraphics[width=0.5\linewidth]{image3.png}
\end{figure}

\section{Bibliografía}

\begin{itemize}
        \item https://programacionymas.com/blog/consumir-una-api-usando-retrofit
        \item https://ljcl79.medium.com/construyendo-una-api-con-node-express-mongodb-f5c9ec86b908
        \item https://es.stackoverflow.com/questions/578339/conexión-a-una-base-de-datos-con-mongodb-y-express-js
        \item https://ironpdf.com/es/java/blog/java-help/okhttp-java/
        \item https://medium.com/@nikita.o.blud/getting-started-with-okhttp3-for-java-http-request-examples-f4623188970e
        \item https://es.stackoverflow.com/questions/106768/cómo-modificar-un-textview-en-android-a-través-de-un-botón
        
    \end{itemize}

\end{document}

